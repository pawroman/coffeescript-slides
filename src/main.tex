%------------------------------------------------------------------------------
%
%   CoffeeScript slides
%       by: Piotr Klibert and Paweł Romanowski
%   
%   Released under the CC BY-SA 3.0 license.
%       See README file for more information.
%
%   Visit github page:
%       https://github.com/promanow/coffeescript-slides/
%
%   And our employer's site:
%       http://10clouds.com
%
%------------------------------------------------------------------------------

%%
%% Setup boilerplate
%%

% beamer stuff
\documentclass[xcolor=dvipsnames]{beamer} 

% lol, this is a beamer command actually - gets rid of bottom navigation bar
\beamertemplatenavigationsymbolsempty

% beamer themes
\usetheme{Rochester}
\usecolortheme[named=MidnightBlue]{structure}

% input & encoding
\usepackage[utf8]{inputenc}
\usepackage[T1]{fontenc}

% other stuff
\usepackage{minted}     % uses pygments to colorize source code
\usepackage{makeidx}    % make index during compilation
\usepackage{hyperref}   % handles links & urls
\hypersetup{
    colorlinks=true,
    linkcolor=black,
    urlcolor=MidnightBlue
}

% MAKE ME INDEX NAO
\makeindex

% coding, with style. ($ pygmentize -L styles)
\usemintedstyle{tango}

% meta info
\title{CoffeeScript}
\subtitle{The little language to make JS better}
\author{Piotr Klibert, Paweł Romanowski}

% command defines

% \slide{title} -> quick way to make a new slide (beamer frame)
% still... need to end it with \end{frame}
\newcommand{\slide}[1]{\begin{frame}[fragile]{{#1}}}

% \coffee -> starts minted with \begin{minted}[gobble=4]{coffeescript}
% still... need to end it with \end{minted}
\newcommand{\coffee}{\begin{minted}[gobble=4]{coffeescript}}

\newcommand{\js}{\begin{minted}[gobble=4]{javascript}}

\newcommand{\epic}[1]{\textbf{\huge{{#1}}}}

% Note: use *SPACE* to indent the code!
% vim: :set expandtab
%      :retab


%%
%% document body
%%

\begin{document}

\begin{frame}[plain]
    \begin{center}
        % coffeescript logo ;)
        \includegraphics[scale=0.25]{img/logo.png}
    \end{center}

    \titlepage

    \begin{center}
        \includegraphics[scale=0.5]{img/cc.png} \\
        Released under the CC BY-SA 3.0 license. \\
        \vspace{0.5cm}
        \url{https://github.com/promanow/coffeescript-slides/}
    \end{center}

\end{frame}

\begin{frame}{What is it about?}
    \tableofcontents
\end{frame}


%%
%% Enough boilerplate! Lets start this shit up 
%%

\section{JavaScript}
%==================================


\subsection{What is JS?}
%----------------------------------
\slide{What is JS?}
    \begin{itemize}
        \item higher order functions
        \item lexically scoped lambdas
        \item dynamic typing, dynamic context (\textbf{this})
        \item object model based on prototypes (BTW: Io language, also Self, LPC)
        \item variable number of function arguments
        \item callbacks, events, asynchronicity

        \pause
        \vspace{0.5cm}

        \item in short: modern, cool higher-level programming language
        \item \ldots used widely
    \end{itemize}
\end{frame}


\subsection{Bad things about JS}
%----------------------------------

\slide{Really, why does it suck?}

    \begin{itemize}
        \item C-like, verbose and bloated syntax
        % verb stays for "verbatim" - and + is in-line delimiter for it
        \item \verb+with+ statement, fallthrough \verb+switch+
        \item broken exception handling (can't specify which exceptions)
        \item differences between implementations
        \item \verb+typeof+ operator
        \item painful debugging
        \item semicolor insertion
        \item lack of dictionaries (maps, associative arrays)
    \end{itemize}
\end{frame}


\slide{Really, why does it suck? (pt. 2)}
    \begin{itemize}
        \item no modules, implicit global scope of variables
        \item Java-like distinction between objects and primitives
        \item new block $\not=$ new scope   % $ is for inline math - pretty cool

        \pause

        \item \epic{type coercion rules}
            \begin{itemize}
                \item \verb+==+ operator
                \item counter-logic casts and promoting rules
                \item \verb+===+ / \verb+!==+ syntax is just stupid
                \item \url{http://wtfjs.com} if you're not convinced yet
            \end{itemize}
        \item and so on \ldots
    \end{itemize}
\end{frame}


\subsection{Attempts to make it better}
%----------------------------------

\slide{Attempts to make JS better}

    \begin{itemize}
        \item libraries
            \begin{itemize}
                \item general: Underscore.js, Functional.js, \ldots
                \item DOM related: jQuery, MooTools, Prototype.js, Dojo, \ldots
                \item OOP related: Class.js, Base.js, Backbone.js, \ldots
            \end{itemize}

        \pause

        \item they all are quite good
        \item but we'd like some more
        \item it's the core syntax that fails
    \end{itemize}

\end{frame}


%======================================
\section{CoffeeScript}
%======================================

\subsection{Intro}

\slide{Enter CoffeeScript}

%% 2-column test
%\begin{columns}[t]
%% remember \coffeescript command gobbles 4 spaces only
%\column{5.5cm}
%    Some CoffeeScript code
%    \coffee
%    class D
%        constructor: (@x) ->
%            @a = 5
%            alert("ALE FAJNE")
%    \end{minted}
%
%\column{5.5cm}
%    Some JavaScript code
%    \js
%    a = 5
%    "Some JS CODE!"
%    \end{minted}
%
%\end{columns}
    \begin{itemize}
        \item an attempt to make working with JS pleasant
        \item WITHOUT changing language semantics (GWT, Objective-J, others)
        \item Python- and Ruby-flavored syntax
        \item ``It's just JavaScript!'' - readable and linted at that
        \item encourages ``best practices'' of JS
        \item written itself in CoffeeScript
        \item can compile CS$\to$JS in the browser
    \end{itemize}

\end{frame}


\slide{Prerequisites}

    \begin{itemize}
        \item \textbf{understanding and respect for JS semantics}
        \item conviction that readability counts
        \item belief that syntax matters
        \item courage to try sth new
    \end{itemize}

\end{frame}


\subsection{Integration}
%----------------------------------
\slide{Integration}

    \begin{itemize}
        \item works extremely well with Underscore and Backbone
        \item in fact, works well with every JS library in existence
        \item node + coffee for quickly trying some code
        \item good support for CS in IDEs (Vim, ofc, rulez!)
        \item out-of-the-box, transparent support in Django-pipeline
    \end{itemize}

\end{frame}



\subsection{Quick syntax tutorial}
%----------------------------------

\slide{Syntax tutorial: basics}
    
    \coffee
    clouds = 10
    ten = "Clouds"
    awesome = true

    # obiekty (indent matters!)
    theCompany =
        name: "10Clouds"
        andNot: "10 clouds"
    # albo tak
    theCompany = {name: "10Clouds", andNot: "10 clouds"}

    list = [1, 2, -666]     # array'e

    # array comprehensions
    cubes = (math.cube num for num in list)

    \end{minted}
\end{frame}

\slide{Iteration}
    
    \coffee
    theCompany =
        name: "10Clouds"
        andNot: "10 clouds"

    # awesomeness
    description = for k, v of theCompany
        "#{k}: #{v}"

    \end{minted}
\end{frame}

\slide{Functions}
    
    \coffee
    a = ->
        5

    alert a()    # 5!
    \end{minted}
\end{frame}

\slide{Classes, @ = this}
    
    \coffee
    class C
        constructor: (@x = 42) ->

    c = new C
    c.x     # 42

    cc = new C 55
    cc.x     # 55

    \end{minted}
\end{frame}


\subsection{Things it does well}
%----------------------------------
\slide{Things CoffeeScript does well}
    \begin{enumerate}
        \item it's just JS with terser syntax
        \item fixes some broken JS syntax
        \item typically $\frac{1}{3}$ less code to write
        \item exposes ``good parts'' of JS
        \item hides ``bad parts'' at the same time
        \item no \emph{runtime} performance penalty
        \item integrates well
    \end{enumerate}
\end{frame}


\subsection{What's not so cool}
%----------------------------------
\slide{What's not so cool?}
    \begin{enumerate}
        \item debugging - we debug result JS
        \item not a separate language
            \begin{itemize}
                \item requires JS understanding
            \end{itemize}
    \end{enumerate}
\end{frame}


%======================================
\section{Further reading}
%======================================
\slide{Further reading}
    \begin{itemize}
        \item Official site - learn by example \\
            \url{http://coffeescript.org/}
        \item The Little Book on CoffeeScript \\
            \url{http://arcturo.github.com/library/coffeescript}
        \item CoffeeScript, Meet Backbone.js: A Tutorial \\
            \url{http://adamjspooner.github.com/coffeescript-meet-backbonejs/}
    \end{itemize}

    Other Coffee-based languages to check out
    \begin{itemize}
        \item coco - CoffeeScript meets Perl and Haskell \\
            \url{https://github.com/satyr/coco}
        \item CoffeeKup - CS based HTML markup \\
            \url{http://coffeekup.org/}
        \item \ldots
    \end{itemize}
\end{frame}


\slide{End}
    \begin{center}
        Thanks for your attention.
        
        \vspace{2 cm}
        \pause
        Questions?
    \end{center}
\end{frame}

\end{document}
