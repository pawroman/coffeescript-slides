%------------------------------------------------------------------------------
%
%   CoffeeScript slides
%       by: Piotr Klibert and Paweł Romanowski
%   
%   Released under the CC BY-SA 3.0 license.
%       See README file for more information.
%
%   Visit github page:
%       https://github.com/promanow/coffeescript-slides/
%
%   And our employer's site:
%       http://10clouds.com
%
%------------------------------------------------------------------------------

%%
%% Setup boilerplate
%%

% beamer stuff
\documentclass[xcolor=dvipsnames]{beamer} 

% lol, this is a beamer command actually - gets rid of bottom navigation bar
\beamertemplatenavigationsymbolsempty

% beamer themes
\usetheme{Rochester}
\usecolortheme[named=MidnightBlue]{structure}

% input & encoding
\usepackage[utf8]{inputenc}
\usepackage[T1]{fontenc}

% other stuff
\usepackage{minted}     % uses pygments to colorize source code
\usepackage{makeidx}    % make index during compilation
\usepackage{hyperref}   % handles links & urls
\hypersetup{
    colorlinks=true,
    linkcolor=black,
    urlcolor=MidnightBlue
}

% MAKE ME INDEX NAO
\makeindex

% coding, with style. ($ pygmentize -L styles)
\usemintedstyle{tango}

% meta info
\title{CoffeeScript}
\subtitle{The little language to make JS better}
\author{Piotr Klibert, Paweł Romanowski}

% command defines

% \slide{title} -> quick way to make a new slide (beamer frame)
% still... need to end it with \end{frame}
\newcommand{\slide}[1]{\begin{frame}[fragile]{{#1}}}

% \coffee -> starts minted with \begin{minted}[gobble=4]{coffeescript}
% still... need to end it with \end{minted}
\newcommand{\coffee}{\begin{minted}[gobble=4]{coffeescript}}

\newcommand{\js}{\begin{minted}[gobble=4]{javascript}}

\newcommand{\epic}[1]{\textbf{\huge{{#1}}}}

% Note: use *SPACE* to indent the code!
% vim: :set expandtab
%      :retab


%%
%% document body
%%

\begin{document}

\begin{frame}[plain]
    \begin{center}
        % coffeescript logo ;)
        \includegraphics[scale=0.25]{img/logo.png}
    \end{center}

    \titlepage

    \begin{center}
        \includegraphics[scale=0.5]{img/cc.png} \\
        Released under the CC BY-SA 3.0 license. \\
        \vspace{0.5cm}
        \url{https://github.com/promanow/coffeescript-slides/}
    \end{center}

\end{frame}

\begin{frame}{What is it about?}
    \tableofcontents
\end{frame}


%%
%% Enough boilerplate! Lets start this shit up 
%%

\section{JavaScript}
%==================================


\subsection{What is JS?}
%----------------------------------
\slide{What is JS?}
    \begin{itemize}
        \item higher order functions
        \item lexically scoped lambdas
        \item dynamic typing, dynamic context (\textbf{this})
        \item object model based on prototypes (BTW: Io language, also Self, LPC)
        \item variable number of function arguments
        \item callbacks, events, asynchronicity

        \pause
        \vspace{0.5cm}

        \item in short: modern, cool higher-level programming language
        \item \ldots used widely
    \end{itemize}
\end{frame}


\subsection{Bad things about JS}
%----------------------------------

\slide{Really, why does it suck?}

    \begin{itemize}
        \item C-like, verbose and bloated syntax
        % verb stays for "verbatim" - and + is in-line delimiter for it
        \item \verb+with+ statement, fallthrough \verb+switch+
        \item broken exception handling (can't specify which exceptions)
        \item differences between implementations
        \item \verb+typeof+ operator
        \item painful debugging
        \item semicolor insertion
        \item lack of dictionaries (maps, associative arrays)
    \end{itemize}
\end{frame}


\slide{Really, why does it suck? (pt. 2)}
    \begin{itemize}
        \item no modules, implicit global scope of variables
        \item Java-like distinction between objects and primitives
        \item new block $\not=$ new scope   % $ is for inline math - pretty cool

        \pause

        \item \epic{type coercion rules}
            \begin{itemize}
                \item \verb+==+ operator
                \item counter-logic casts and promoting rules
                \item \verb+===+ / \verb+!==+ syntax is just stupid
                \item \url{http://wtfjs.com} if you're not convinced yet
            \end{itemize}
        \item and so on \ldots
    \end{itemize}
\end{frame}


\subsection{Attempts to make it better}
%----------------------------------

\slide{Attempts to make JS better}

    \begin{itemize}
        \item libraries
            \begin{itemize}
                \item general: Underscore.js, Functional.js, \ldots
                \item DOM related: jQuery, MooTools, Prototype.js, Dojo, \ldots
                \item OOP related: Class.js, Base.js, Backbone.js, \ldots
            \end{itemize}

        \pause

        \item they all are quite good
        \item but we'd like some more
        \item it's the core syntax that fails
    \end{itemize}

\end{frame}


%======================================
\section{CoffeeScript}
%======================================

\subsection{Intro}

\slide{Enter CoffeeScript}

%% 2-column test
%\begin{columns}[t]
%% remember \coffeescript command gobbles 4 spaces only
%\column{5.5cm}
%    Some CoffeeScript code
%    \coffee
%    class D
%        constructor: (@x) ->
%            @a = 5
%            alert("ALE FAJNE")
%    \end{minted}
%
%\column{5.5cm}
%    Some JavaScript code
%    \js
%    a = 5
%    "Some JS CODE!"
%    \end{minted}
%
%\end{columns}
    \begin{itemize}
        \item an attempt to make working with JS pleasant
        \item WITHOUT changing language semantics (GWT, Objective-J, others)
        \item Python- and Ruby-flavored syntax over C-flavor
        \item motto: ``It's just JavaScript!'' - readable and linted at that
        \item encourages ``best practices'' of JS
        \item written itself in CoffeeScript
        \item can compile CS$\to$JS in the browser on the fly
    \end{itemize}

\end{frame}


\slide{Prerequisites}

    \begin{itemize}
        \item \textbf{understanding and respect for JS semantics}
        \item conviction that readability counts
        \item belief that syntax matters
        \item courage to try sth new
    \end{itemize}

\end{frame}


\subsection{Integration}
%----------------------------------
\slide{Integration}

    \begin{itemize}
        \item works extremely well with Underscore and Backbone
        \item in fact, works well with every JS library in existence
        \item node + coffee for quickly trying some code
        \item fast compilation
        \item good support for CS in IDEs (Vim, ofc, rulez!)
        \item out-of-the-box, transparent support in Django-pipeline
    \end{itemize}

\end{frame}



\subsection{Quick syntax tutorial}
%----------------------------------

\slide{(Quick) syntax tutorial: basics}

\begin{block}{Note}
    For the sake of debugging, it's important to know what your code compiles to.
\end{block}

    \coffee
    # functions
    simpleFunc = ->
        5   # implicit return

    # works similar to *args in Python
    func = (args...) ->
        console.log arg for arg in args 
        null

    # can call functions without braces
    func "some", "args", "for", "func"
    func ["some", "args", "for", "func"]... # unpack
    \end{minted}

\end{frame}


\slide{Syntax tutorial: default arguments}

    \coffee
    pow = (x, p = 2) ->
        ret = 1
        for i in [0..p]  # range
            ret *= x
        ret

    pow 9  # 9 squared
    pow(9, 3)  # 9 cubed
    \end{minted}

\end{frame}


\slide{Syntax tutorial: strings}

    \coffee
    s = "Multiline strings work
            as in Python!" # indents converted to spaces

    s2 = """A string without indents
            stripped"""    # will contain \n etc

    # string interpolation
    templ = (ctx) ->
        "I would like some #{ctx.what} please!"

    # call templ
    msg = templ({what: "Coffee"})

    # CoffeeScript's fancy way
    msg = templ
        what: "Coffee"
    \end{minted}

\end{frame}


\slide{More on strings and function calling}

    \coffee
    templ = (ctx) ->
        "I would like some #{ctx.what} please!"

    console.log templ {what: "Coffee"}
    console.log templ
        what: "Coffee"

    # These are all valid:
    console.log templ({what: "Coffee"})
    console.log (templ {what: "Coffee"})
    console.log(templ {what:"Coffee"})

    # there could be anything in #{}
    "counting: #{x for x in [1..10]}"
    "side effect of interpolation: #{console.log [1..10]}"
    \end{minted}

\end{frame}


\slide{Iteration, comprehensions}
    
    \coffee
    numbers = (x for x in [1..10])  # [1, ..., 10]
    # every loop returns a list of values from each step
    numbers = for x in [1..10]
        x

    numbers = [1..10]  # inclusive
    numbers2 = [1...10]  # exclusive

    # slices
    slice = numbers[3...5] # equivalent of Python's [3:5]

    # destructuring assignment (should be in the future JS)
    [first, middle..., last] = [1..10]

    first == 1 and last == 10 and middle.length == 8
    \end{minted}

\end{frame}


\slide{Iteration, comprehensions, pt.2}
    
    \coffee
    obj = 
        property: "value"
        number: 9

    # equiv. of:
    obj = {
        "property" : "value",
        "number" : 9
    }

    for k, v of obj  # note the "of"!
        console.log k, v

    # keys only
    for k of obj
        console.log k
    \end{minted}

\end{frame}


\slide{Classes, @ = this}

    \begin{block}{Note}
        Have you read Backbone.js sources? You really should - \verb+class+ and 
        \verb+extends+ keywords implement the same inheritance model that
        \verb+Backbone.View.extends+ provides.
    \end{block}
    
    \coffee
    class C
        # default value with assignment
        constructor: (@x = 42) ->   # empty func. body
        getX: ->
            @x

    c = new C
    c.getX()     # 42, needs ()!

    cc = new C 55
    cc.x     # 55
    \end{minted}

\end{frame}


\slide{Classes, inheritance}
    \coffee
    class Base
        constructor: (args...) ->
            @args = args

    class Sub extends Base
        constructor: (args...) ->
            # :: stands for .prototype
            args = (String::toUpperCase.call(x) for x in args)
            super(args...)  # unpack args

    sub = new Sub("some", "silly", "strings")

    sub instanceof Sub  # true
    sub instanceof Base  # also true! for free!
    \end{minted}
\end{frame}


\slide{Classes: @ binding via => (and Backbone integration)}
    \coffee
    class MyView extends Backbone.View
        events:
            "click .my-button" : "buttonClicked"

        initialize: () ->
            # => arrow defines a function that
            # binds @ (this)
            @el.find(".second-button").bind 'click', (ev) =>
                @secondClicked = true
        
        buttonClicked: (ev) =>
            # automatically bound to each instance 
            @buttonClicked = true
    \end{minted}
\end{frame}


\slide{Other syntax sugar}
    \coffee
    # chained comparisons
    a = if 0 < i < 10 then i else "Not in range!"

    # existential operator and assignment
    if not variable?
        variable = "default value"

    # shorter form
    variable ?= "default value"

    # also in place of accessor `.`
    obj = {property: "value"}
    obj.property.toUpperCase()  # ok
    obj.otherProperty.toUpperCase()  # TypeError!
    obj.otherProperty?.toUpperCase()  # undefined
    \end{minted}
\end{frame}


\slide{Other syntax sugar pt.2}

    \coffee
    # fixed "switch"
    switch command
        when "work" then doWork()
        when "chill", "relax"
            goToHawaii "now"
        else doNothing()

    # using destructuring to swap
    x = -666
    y = 0

    [x, y] = [y, x]
    \end{minted}
\end{frame}


\subsection{Things it does well}
%----------------------------------
\slide{Things CoffeeScript does well}
    \begin{enumerate}
        \item it's just JS with terser syntax
        \item fixes some broken JS syntax
        \item typically $\frac{1}{3}$ less code to write
        \item exposes ``good parts'' of JS
        \item hides ``bad parts'' at the same time
        \item no \emph{runtime} performance penalty
        \item integrates well
        \item stable, production-ready
        \item fun to write, appeals to Pythonistas ;)
    \end{enumerate}
\end{frame}


\subsection{What's not so cool}
%----------------------------------
\slide{What's not so cool?}
    \begin{enumerate}
        \item debugging - we debug result JS
        \item not a separate language
            \begin{itemize}
                \item requires JS understanding
                \item won't fix everything
                \item sometimes strange things may happen (it's JS after all)
            \end{itemize}
    \end{enumerate}
\end{frame}


%======================================
\section{Further reading}
%======================================
\slide{Further reading}
    \begin{itemize}
        \item Official site - learn by example + side-by-side comparisons \\
            \url{http://coffeescript.org/}
        \item The Little Book on CoffeeScript \\
            \url{http://arcturo.github.com/library/coffeescript}
        \item CoffeeScript, Meet Backbone.js: A Tutorial \\
            \url{http://adamjspooner.github.com/coffeescript-meet-backbonejs/}
    \end{itemize}

    Other Coffee-based languages to check out
    \begin{itemize}
        \item coco - CoffeeScript meets Perl and Haskell \\
            \url{https://github.com/satyr/coco}
        \item CoffeeKup - CS based HTML markup \\
            \url{http://coffeekup.org/}
        \item \ldots
    \end{itemize}
\end{frame}


\slide{End}
    \begin{center}
        Thanks for your attention.
        
        \vspace{2 cm}
        \pause
        Questions?
    \end{center}
\end{frame}

\end{document}
