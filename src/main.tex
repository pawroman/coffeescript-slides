%------------------------------------------------------------------------------
%
%   CoffeeScript slides
%       by: Piotr Klibert and Paweł Romanowski
%   
%   Released under the CC BY-SA 3.0 license.
%       See README file for more information.
%
%   Visit github page:
%       https://github.com/promanow/coffeescript-slides/
%
%   And our employer's site:
%       http://10clouds.com
%
%------------------------------------------------------------------------------

%%
%% Setup boilerplate
%%

% beamer stuff
\documentclass[xcolor=dvipsnames]{beamer} 

% lol, this is a beamer command actually - gets rid of bottom navigation bar
\beamertemplatenavigationsymbolsempty

% beamer themes
\usetheme{Rochester}
\usecolortheme[named=MidnightBlue]{structure}

% input & encoding
\usepackage[utf8]{inputenc}
\usepackage[T1]{fontenc}

% other stuff
\usepackage{minted}     % uses pygments to colorize source code
\usepackage{makeidx}    % make index during compilation
\usepackage{hyperref}   % handles links & urls
\hypersetup{
    colorlinks=true,
    linkcolor=black,
    urlcolor=MidnightBlue
}

% MAKE ME INDEX NAO
\makeindex

% coding, with style. ($ pygmentize -L styles)
\usemintedstyle{tango}

% meta info
\title{CoffeeScript}
\subtitle{XXX: invent the title :P}
\author{Piotr Klibert, Paweł Romanowski}

% command defines

% \slide{title} -> quick way to make a new slide (beamer frame)
% still... need to end it with \end{frame}
\newcommand{\slide}[1]{\begin{frame}[fragile]{{#1}}}

% \coffee -> starts minted with \begin{minted}[gobble=4]{coffeescript}
% still... need to end it with \end{minted}
\newcommand{\coffee}{\begin{minted}[gobble=4]{coffeescript}}

\newcommand{\js}{\begin{minted}[gobble=4]{javascript}}

\newcommand{\epic}[1]{\textbf{\huge{{#1}}}}

% Note: use *SPACE* to indent the code!
% vim: :set expandtab
%      :retab


%%
%% document body
%%

\begin{document}

\begin{frame}[plain]
    \begin{center}
        % coffeescript logo ;)
        \includegraphics[scale=0.25]{img/logo.png}
    \end{center}

    \titlepage

    \begin{center}
        \includegraphics[scale=0.5]{img/cc.png} \\
        Released under the CC BY-SA 3.0 license. \\
        \vspace{0.5cm}
        \url{https://github.com/promanow/coffeescript-slides/}
    \end{center}

\end{frame}

\begin{frame}{What is it about?}
    \tableofcontents
\end{frame}


%%
%% Enough boilerplate! Lets start this shit up 
%%

\section{JavaScript}
%==================================

\subsection{Is JavaScript really that bad?}
%----------------------------------
\begin{frame}[fragile]{}
    Is JavaScript really \textbf{that} bad?
\end{frame}

\slide{There are many reasons to hate JS}
    Most of them are either outdated, not JS fault or completely subjective. \\ % blank line
    
    \pause

    \begin{itemize}
        \item lack of dictionaries (maps, associative arrays)
        \item nested blocks $\not=$ nested scopes
        \item everything in JavaScript is mutable (well, strings are not.)
        \item differences between implementations (between browsers - before jQuery)
        \item it is extremely slow (it is most certainly NOT slow)
    \end{itemize}
\end{frame}


\slide{...But some of them are by design!}

    \begin{itemize}
        \item \textbf{this} - dynamic context of execution
        \item (fake OOP) prototypal inheritence (not class-based one)
        \item callbacks for everything; everything is asynchronous (and event based)
    \end{itemize}
\end{frame}


\subsection{JavaScript fails}
%----------------------------------
\slide{But there are real fails, aren't there?}
    Oh, and how many! \\

    \begin{itemize}
        % verb stays for "verbatim" - and + is in-line delimiter for it
        \item \verb+while+ statement - so broken noone uses it
        \item implicit global scope of variables
        \item no modules of any kind
        \item semicolor insertion
        \item painful debugging
        \pause
        \item \epic{type coercion rules}
            \begin{itemize}
                \item counter-logic casts and promoting rules
                \item \verb+===+ / \verb+!==+ syntax is just stupid
            \end{itemize}
    \end{itemize}
\end{frame}


\subsection{JS alternatives}
%----------------------------------

\slide{JS alternatives}
    There are some: \\

    \begin{itemize}
        \item Libraries to make life easier
            \begin{itemize}
                \item jQuery
                \item ...?
            \end{itemize}
        \pause   % make a pause - new slide
        \item Languages compiled to JS
            \begin{itemize}
                \item some lang
                \item CoffeeScript!!1
            \end{itemize}
    \end{itemize}

\end{frame}


%======================================
\section{CoffeeScript}
%======================================

\subsection{Intro}

\slide{Intro}

% 2-column test
\begin{columns}[t]
% remember \coffeescript command gobbles 4 spaces only
\column{5cm}
    Some CoffeeScript code
    \coffee
    class D
        constructor: (@x) ->
            @a = 5
            alert("ALE FAJNE")
    \end{minted}

\column{5cm}
    Some JavaScript code

    \js
    a = 5
    "Some JS CODE!"
    \end{minted}

\end{columns}

\end{frame}

\subsection{Quick CoffeeScript syntax tutorial}
%----------------------------------

\slide{Basics - variables, built-in types}
    
    \coffee
    clouds = 10
    ten = "Clouds"
    awesome = true

    # obiekty (indent matters!)
    theCompany =
        name: "10Clouds"
        andNot: "10 clouds"
    # albo tak
    theCompany = {name: "10Clouds", andNot: "10 clouds"}

    list = [1, 2, -666]     # array'e

    # array comprehensions
    cubes = (math.cube num for num in list)

    \end{minted}
\end{frame}

\slide{Iteration}
    
    \coffee
    theCompany =
        name: "10Clouds"
        andNot: "10 clouds"

    # awesomeness
    description = for k, v of theCompany
        "#{k}: #{v}"

    \end{minted}
\end{frame}

\slide{Functions}
    
    \coffee
    a = ->
        5

    alert a()    # 5!
    \end{minted}
\end{frame}

\slide{Classes, @ = this}
    
    \coffee
    class C
        constructor: (@x = 42) ->

    c = new C
    c.x     # 42

    cc = new C 55
    cc.x     # 55

    \end{minted}
\end{frame}


\subsection{Things CoffeeScript does better}
%----------------------------------
\slide{Things CoffeeScript does better}
    \begin{enumerate}
        \item Fixes some broken JS syntax
        \item Integrates with Backbone classes nicely
        \item ...
    \end{enumerate}
\end{frame}


\subsection{What's not so cool}
%----------------------------------
\slide{What's not so cool?}
    \begin{enumerate}
        \item Debugging - we debug result JS
        \item ...
    \end{enumerate}
\end{frame}


%======================================
\section{Further reading}
%======================================
\slide{Further reading}
    \begin{itemize}
        \item Official site - learn by example \\
            \url{http://coffeescript.org/}
        \item The Little Book on CoffeeScript \\
            \url{http://arcturo.github.com/library/coffeescript/index.html}
        \item CoffeeScript, Meet Backbone.js: A Tutorial \\
            \url{http://adamjspooner.github.com/coffeescript-meet-backbonejs/}
    \end{itemize}
\end{frame}


\slide{End}
    \begin{center}
        Thanks for your attention.
        
        \vspace{2 cm}
        \pause
        Questions?
    \end{center}
\end{frame}

\end{document}
