% beamer stuff
\documentclass[xcolor=dvipsnames]{beamer} 
\usetheme{Rochester}
\usecolortheme[named=MidnightBlue]{structure}

% bądź polski.
\usepackage{polski}
\usepackage[polish]{babel}
\usepackage[utf8]{inputenc}

% inne pakiety
\usepackage{minted}     % używa pygments do kolorowania kodu
\usepackage{makeidx}    % do generacji "indeksu" - potrzebne do spisu treści

% MAKE ME INDEX NAO
\makeindex

% coding, with style. ($ pygmentize -L styles)
\usemintedstyle{trac}

% meta info
\title{CoffeeScript}
\subtitle{Prezentacja - cośtam}
\author{Piotr Klibert, Paweł Romanowski}

% document body
\begin{document}

\begin{frame}[plain]
    \begin{center}
        \includegraphics[scale=0.25]{img/logo.png}
    \end{center}
    \titlepage
\end{frame}

\begin{frame}{O czym będzie?}
    \tableofcontents
\end{frame}

%======================================
\section{Wstęp}
%======================================

\subsection{Czemu JS jest ble?}

\begin{frame}[fragile]{Czemu JS jest ble?}
    \begin{itemize}
        \item bo a
        \item bo b
    \end{itemize}
\end{frame}

\subsection{Alternatywy do JS}
\begin{frame}[fragile]{Alternatywy}
    Alternatywy?

    \begin{itemize}
        \item jQuery
        \pause   % zrób pauzę - nowy slajd
        \item CoffeeScript!!1
    \end{itemize}
\end{frame}

%======================================
\section{bLA}
%======================================

\subsection{ble}

% musi być fragile żeby działał minted...
\begin{frame}[fragile]{Slajd2}
    Trochę kodu.

    % używać spacji do indentu kodu!
    % vim: :set expandtab
    % jeśli nadal nie działa: :retab
    % gobble=4 żeby zeżarł 4 poprzedzające spacje
    \begin{minted}[gobble=4]{coffeescript}
    class D
        constructor: (@x) ->
            @a = 5
            alert("ALE FAJNE")
    \end{minted}
\end{frame}

%======================================
\section{Przyspieszony kurs składni}
%======================================

\begin{frame}[fragile]{Podstawy}
    
    \begin{minted}[gobble=4]{coffeescript}
    clouds = 10
    ten = "Clouds"
    awesome = true

    # obiekty (indent matters!)
    theCompany =
        name: "10Clouds"
        andNot: "10 clouds"
    # albo tak
    theCompany = {name: "10Clouds", andNot: "10 clouds"}

    list = [1, 2, -666]     # array'e

    # array comprehensions
    cubes = (math.cube num for num in list)

    \end{minted}
\end{frame}

\begin{frame}[fragile]{Iteracja}
    
    \begin{minted}[gobble=4]{coffeescript}
    theCompany =
        name: "10Clouds"
        andNot: "10 clouds"

    # awesomeness
    description = for k, v of theCompany
        "#{k}: #{v}"

    \end{minted}
\end{frame}

\begin{frame}[fragile]{Funkcje}
    
    \begin{minted}[gobble=4]{coffeescript}
    a = ->
        5

    alert a()    # 5!
    \end{minted}
\end{frame}

\begin{frame}[fragile]{Klasy, @ = this}
    
    \begin{minted}[gobble=4]{coffeescript}
    class C
        constructor: (@x = 42) ->

    c = new C
    c.x     # 42

    cc = new C 55
    cc.x     # 55

    \end{minted}
\end{frame}

\begin{frame}[fragile]{Koniec}
    \begin{center}
        Dziękujemy za uwagę.
        
        \vspace{2 cm}
        \pause
        Pytania?
    \end{center}
\end{frame}

\end{document}
